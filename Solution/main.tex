\documentclass[11pt]{article}

\usepackage{amsmath}
\usepackage{amssymb}
\usepackage{amsthm}
\usepackage{hyperref}
\usepackage{ulem}
\usepackage{enumitem}
\usepackage[left=0.75in, right=0.75in, bottom=0.75in]{geometry}
\usepackage{graphicx}
\usepackage{float}
\usepackage{fancyhdr}
\pagestyle{fancy}
\fancyhf{}
\lhead{190100044 \& 190100055}
\rhead{CS 215: Assignment 1}
\renewcommand{\footrulewidth}{1.0pt}
\cfoot{Page \thepage}

\setlength{\parindent}{0em}

\title{Assignment 1: CS 215}
\author{190100044 \& 190100055}
\date{31st August 2020}

\begin{document}

\maketitle
\tableofcontents
\thispagestyle{empty}

\newpage \setcounter{page}{1}
\section{Question 1}
\begin{enumerate}[label=(\alph*)]
    \item The probability that the first person picks up his/her own book is $\frac{1}{n}$, for the second person, there are $n-1$ books remaining, thus the probability for him/her is $\frac{1}{n-1}$, similarly, the $i^{th}$ person has $ (n-i+1) $ choices, out of which $1$ is correct, thus $P(E_i)$, the probability that the $i^{th}$ person picks up his/her own book is given by: $\frac{1}{n-i+1}$.\\
          Thus, the probability that each person picks up his/her own book ($ \cap_{i=1}^{n} P(E_i) $) is given by:
          $$ \prod_{i=1}^n \Big( \frac{1}{n-i+1} \Big) = \frac{1}{n!}$$
          Alternative:\\
          $N(\text{Sample space}) = n!$ and only one unique permutation is possible such that every person picks his/her book. Therefore,
          $$ P(E) = \frac{1}{n!} $$

    \item Similar to the previous case, we have, the probability that the first $ m < n $ persons who picked up a book receive their own book back again ($ \cap_{i=1}^{m} P(E_i) $),
          $$ \prod_{i=1}^m \Big( \frac{1}{n-i+1} \Big) = \frac{(n-m)!}{n!} $$

    \item The first person to pick a book now has $m$ correct choices out of $n$, similarly, the second person has $m-1$ correct choices out of $n-1$ books. Thus, the $i^{th}$ person ($ i \le m $) has $ (m-i+1) $ correct choices, out of $(n-i+1)$ choices. Thus $P(E_i)$, the probability that the $i^{th}$ person ($i < m$) gets back a book belonging to \textbf{one} of the last $m$ persons is equal to: $\frac{m-i+1}{n-i+1}$.\\
          Thus, the probability that each person among the first $m$ persons to pick up the book gets back a book belonging to one of the last $m$ persons to pick up the books is: ($ \cap_{i=1}^{m} P(E_i) $)
          $$ \prod_{i=1}^m \Big( \frac{m-i+1}{n-i+1} \Big) = \frac{m!(n-m)!}{n!} $$

    \item The first $m$ persons can pick up any book and each book needs to be clean. Thus $P(E_i)$, the probability that the $i^{th}$ person picks up a book which is clean, is equal to $(1-p)$.\\
          Thus, the probability that each person among the first $m$ persons picks up a clean book is: ($ \cap_{i=1}^{m} P(E_i) $)
          $$ (1-p)^m $$

    \item Exactly $m$ persons\footnote{Note that these $m$ persons \textbf{may not} be the first $m$ persons} need to pick up a clean book.\\
          Thus, the ways to choose $m$ persons out of $n$ is ${n\choose m}$, all these chosen people must have chosen a clean book (probability: ($1-p$)), while all the ($ n-m $) remaining people must choose an unclean book (probability: $p$), since it is mentioned \textbf{\textit{exactly}} $m$ persons must choose a clean book.\\
          Thus this probability is given by:
          $$ {n\choose m} (1-p)^m (p)^{n-m} $$

\end{enumerate}

\newpage
\section{Question 2}
By the definition of the variance\footnote{Note: Here $\sigma$ is defined to be the \textbf{positive} square root of the variance}, $\sigma^2$, we have:
$$ \sigma^2 = \frac{\sum_{i=1}^n (x_i - \mu)^2}{n-1} $$
\begin{equation}
    \label{var}
    (n-1)\sigma^2 = \sum_{i=1}^n (x_i - \mu)^2
\end{equation}
Since $(x_i - \mu)^2 \ge 0 \  \forall \ i$, we have ((for all $i$)):
$$ \sum_{i=1}^n (x_i - \mu)^2 \ge (x_i - \mu)^2  $$
Thus, by (\ref{var}) we get (for all $i$):
$$ (n-1)\sigma^2 \ge (x_i - \mu)^2 $$
Taking the square root on both sides,
$$ \sigma\sqrt{n-1} \ge |x_i - \mu| \implies |x_i - \mu| \le \sigma\sqrt{n-1} \hfill $$
\hfill \qedsymbol \\
\noindent
The (Two-sided) Chebyshev's inequality states that:
$$ |S_k| = \{ x_i : \  |x_i - \mu| \ge k\sigma \} $$
\begin{equation}
    \frac{|S_k|}{n} < \frac{1}{k^2}
\end{equation}
Consider $k = \sqrt{n-1}$ in the above equation,\\
$$ |S_{\sqrt{n-1}}| = \{ x_i : \  |x_i - \mu| \ge \sigma\sqrt{n-1} \} $$
$$ \frac{|S_{\sqrt{n-1}}|}{n} < \frac{1}{n-1} $$
$$ 0 \leq |S_{\sqrt{n-1}}| < \frac{n}{n-1}$$

``TODO: How does this inequality compare with Chebyshev's inequality as n increases?"\\
The inequality proven above says that $\forall i \ |x_i - \mu| \le \sigma \sqrt{n-1}$, i.e. $|S_{\sqrt{n-1}}| = 0$.\\
Applying limit $n \to \infty$ and Sandwich Theorem, $\lim_{n\to\infty} |S_{\sqrt{n-1}}| = 0$.

\newpage
\section{Question 3}

\newpage
\section{Question 4}
Let $A$ be the event that the rickshaw is red.\\
Thus, as a given rickshaw must be either red or blue, $A^c$ is the event that the rickshaw is blue.\\
Let $B$ be the event that the rickshaw is observed as red by the person under the given conditions.\\
We have been given the following:
$$
    P(A) = \frac{1}{100} \implies P(A^c) = \frac{99}{100}
$$
$$
    P(B|A) = 0.99 \ \  P(B|A^c) = 0.02
$$
We need to find $P(A|B)$. \\
\\
We shall first find $P(B)$,
$$
    \begin{aligned}
        P(B) & = P(B|A)P(A) + P(B|A^c)P(A^c)     \\
             & = 0.99\times0.01 + 0.02\times0.99 \\
             & = 0.0297
    \end{aligned}
$$
\\
Now, by Bayes Theorem,
$$
    \begin{aligned}
        P(A|B) & = \frac{P(B|A)P(A)}{P(B)}            \\
               & = \frac{0.99\times0.01}{0.0297}      \\
               & = \frac{1}{3} = 0.33 = \mathbf{33\%}
    \end{aligned}
$$
Thus, the defence lawyer can argue that there was only a $33\%$ chance that the rickshaw observed by the person XYZ was indeed red.

\newpage
\section{Question 5}
\begin{enumerate}[label=(\alph*)]
    \item
          Event of choosing a door is independent of which door contains the car as the contestant is unaware of the car's location. Therefore, $C_i$ and $Z_i$ are independent events.
          \begin{equation*} \label{eq1}
              \begin{split}
                  P(C_i|Z_1) &= P(C_i) \hspace{3em} \forall i \in {1, 2, 3} \\
                  &= \mathbf{\frac{1}{3}} \hspace{5em} \forall i \in {1, 2, 3}
              \end{split}
          \end{equation*}

    \item
          Keeping in mind, the fact given about the intelligence of the host, we can produce following conclusions.\\
          If $(C_1, Z_1)$ happen, then the host has equal chances of choosing Door 2 and Door 3, therefore,
          \begin{align*}
              P(H_1|C_1,Z_1) & = 0 & P(H_2|C_1,Z_1) & = \frac{1}{2} & \mathbf{P(H_3|C_1,Z_1)} & = \mathbf{\frac{1}{2}}
          \end{align*}
          If $(C_2, Z_1)$ happen, then the host will open Door 3, therefore,
          \begin{align*}
              P(H_1|C_2,Z_1) & = 0 & P(H_2|C_2,Z_1) & = 0 & \mathbf{P(H_3|C_2,Z_1)} & = \mathbf{1}
          \end{align*}
          If $(C_3, Z_1)$ happen, then the host will open Door 2, therefore,
          \begin{align*}
              P(H_1|C_3,Z_1) & = 0 & P(H_2|C_3,Z_1) & = 1 & \mathbf{P(H_3|C_3,Z_1)} & = \mathbf{0}
          \end{align*}

    \item
          \begin{align*}
              \begin{split}
                  P(H_3, Z_1) &= P(H_3 \cap (C_1 \cup C_2 \cup C_3) \cap Z_1) \\
                  &= P((H_3 \cap (C_1 \cap Z_1)) \cup (H_3 \cap (C_1 \cap Z_1)) \cup (H_3 \cap (C_1 \cap Z_1))) \\
                  &= P(H_3|C_1,Z_1) \cdot P(C_1,Z_1) + P(H_3|C_2,Z_1) \cdot P(C_2,Z_1) + P(H_3|C_2,Z_1) \cdot P(C_2,Z_1) \\
                  &= \frac{1}{2} \cdot \frac{1}{9} + 1 \cdot \frac{1}{9} + 0 \cdot \frac{1}{9} \\
                  &= \frac{1}{6}
              \end{split} \\\\
              \begin{split}
                  P(C_2|H_3, Z_1) &= \frac{P(H_3|C_2, Z_1) \cdot P(C_2, Z_1)}{P(H_3, Z_1)} \\
                  &= \frac{1 \cdot \frac{1}{9}}{\frac{1}{6}} \\
                  &= \frac{2}{3} \\
                  &= \mathbf{66\%}
              \end{split}
          \end{align*}

    \item
          $P(H_3|C_3,Z_1) = 0 \implies P(C_3|H_3, Z_1)=0$ \hspace{1em} (Host is intelligent)\\
          Therefore, $P(C_1|H_3, Z_1) = 1 - P(C_2|H_3, Z_1) = \frac{1}{3} = \mathbf{66\%}$

    \item
          $P(C_2|H_3, Z_1) > P(C_1|H_3, Z_1)$, i.e., the chances of winning the car is higher if the contestant switches his/her choice.

    \item ``TODO: Not sure what is happening here.''

\end{enumerate}

\newpage
\section{Question 6}
\begin{figure}[H]
    \centering
    \includegraphics[scale=0.9]{q6.1.png}
\end{figure}
\begin{figure}[H]
    \centering
    \includegraphics[scale=0.9]{q6.2.png}
\end{figure}
\begin{center}
    \begin{tabular}{|c|c|c|}
        \hline
        \textbf{Results} & \textbf{Mean Square Error (30\% Corruption)} & \textbf{Mean Square Error (60\% Corruption)} \\
        \hline
        Corrupted        & 266.09                                       & 588.35                                       \\
        \hline
        Median           & 26.52                                        & 372.31                                       \\
        \hline
        Mean             & 90.15                                        & 737.09                                       \\
        \hline
        Quartile         & 0.01                                         & 117.38                                       \\
        \hline
    \end{tabular}
\end{center}
\noindent
\textbf{Quartile} method produced the best relative mean squared error.
\subsection*{Explanation:}





\newpage
\section{Question 7}

\end{document}

