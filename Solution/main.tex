\documentclass[11pt]{article}

\usepackage{amsmath}
\usepackage{amssymb}
\usepackage{amsthm}
\usepackage{hyperref}
\usepackage{ulem}
% \usepackage[left=0.5in, right=0.5in, bottom=1in]{geometry}

\usepackage{fancyhdr}
\pagestyle{fancy}
\fancyhf{}
\lhead{190100044 \& 190100055}
\rhead{CS 215: Assignment 1}
\renewcommand{\footrulewidth}{1.0pt}
\cfoot{Page \thepage}

\title{Assignment 1: CS 215}
\author{190100044 \& 190100055}
\date{31st August 2020}

\begin{document}

\maketitle
\tableofcontents
\thispagestyle{empty}

\newpage \setcounter{page}{1}
\section{Question 1}

\newpage
\section{Question 2}
By the definition of the variance\footnote{Note: Here $\sigma$ is defined to be the \textbf{positive} square root of the variance}, $\sigma^2$, we have:
$$ \sigma^2 = \frac{\sum_{i=1}^n (x_i - \mu)^2}{n-1} $$
\begin{equation}
    \label{var}
    (n-1)\sigma^2 = \sum_{i=1}^n (x_i - \mu)^2
\end{equation}
Since $(x_i - \mu)^2 \ge 0 \  \forall \ i$, we have ((for all $i$)):
$$ \sum_{i=1}^n (x_i - \mu)^2 \ge (x_i - \mu)^2  $$
Thus, by (\ref{var}) we get (for all $i$):
$$ (n-1)\sigma^2 \ge (x_i - \mu)^2 $$
Taking the square root on both sides,
$$ \sigma\sqrt{n-1} \ge |x_i - \mu| \Rightarrow |x_i - \mu| \le \sigma\sqrt{n-1} \hfill $$
\hfill \qedsymbol \\
\noindent
The (Two-sided) Chebyshev's inequality states that:
$$ |S_k| = \{ x_i : \  |x_i - \mu| \ge k\sigma \} $$
\begin{equation}
    \frac{|S_k|}{n} < \frac{1}{k^2}
\end{equation}
``TODO: How does this inequality compare with Chebyshev's inequality as n increases?"

\newpage
\section{Question 3}

\newpage
\section{Question 4}
Let $A$ be the event that the rickshaw is red.\\
Thus, as a given rickshaw must be either red or blue, $A^c$ is the event that the rickshaw is blue.\\
Let $B$ be the event that the rickshaw is observed as red by the person under the given conditions.\\
We have been given the following:
$$ 
    P(A) = \frac{1}{100} \Rightarrow P(A^c) = \frac{99}{100}
$$
$$
    P(B|A) = 0.99 \ \  P(B|A^c) = 0.02
$$
We need to find $P(A|B)$. \\
\\
We shall first find $P(B)$,
$$
\begin{aligned}
    P(B) &= P(B|A)P(A) + P(B|A^c)P(A^c)\\
    &= 0.99\times0.01 + 0.02\times0.99\\
    &= 0.0297
\end{aligned}
$$
\\
Now, by Bayes Theorem,
$$
\begin{aligned}
    P(A|B) &= \frac{P(B|A)P(A)}{P(B)}\\
    &= \frac{0.99\times0.01}{0.0297}\\
    &= \frac{1}{3} = 0.33 = 33\%
\end{aligned}
$$
Thus, the defence lawyer can argue that there was only a $33\%$ chance that the rickshaw observed by the person XYZ was indeed red.

\newpage
\section{Question 5}

\newpage
\section{Question 6}

\newpage
\section{Question 7}

\end{document}